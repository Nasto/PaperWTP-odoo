
%% bare_jrnl.tex
%% V1.4
%% 2012/12/27
%% by Michael Shell
%% see http://www.michaelshell.org/
%% for current contact information.

%% Support sites:
%% http://www.michaelshell.org/tex/ieeetran/
%% http://www.ctan.org/tex-archive/macros/latex/contrib/IEEEtran/
%% and
%% http://www.ieee.org/

%%*************************************************************************
%% Legal Notice:
%% This code is offered as-is without any warranty either expressed or
%% implied; without even the implied warranty of MERCHANTABILITY or
%% FITNESS FOR A PARTICULAR PURPOSE! 
%% User assumes all risk.
%% In no event shall IEEE or any contributor to this code be liable for
%% any damages or losses, including, but not limited to, incidental,
%% consequential, or any other damages, resulting from the use or misuse
%% of any information contained here.
%%
%% All comments are the opinions of their respective authors and are not
%% necessarily endorsed by the IEEE.
%%
%% This work is distributed under the LaTeX Project Public License (LPPL)
%% ( http://www.latex-project.org/ ) version 1.3, and may be freely used,
%% distributed and modified. A copy of the LPPL, version 1.3, is included
%% in the base LaTeX documentation of all distributions of LaTeX released
%% 2003/12/01 or later.
%% Retain all contribution notices and credits.
%% ** Modified files should be clearly indicated as such, including  **
%% ** renaming them and changing author support contact information. **
%%
%% File list of work: IEEEtran.cls, IEEEtran_HOWTO.pdf, bare_adv.tex,
%%                    bare_conf.tex, bare_jrnl.tex, bare_jrnl_compsoc.tex,
%%                    bare_jrnl_transmag.tex
%%*************************************************************************

% Note that the a4paper option is mainly intended so that authors in
% countries using A4 can easily print to A4 and see how their papers will
% look in print - the typesetting of the document will not typically be
% affected with changes in paper size (but the bottom and side margins will).
% Use the testflow package mentioned above to verify correct handling of
% both paper sizes by the user's LaTeX system.
%
% Also note that the "draftcls" or "draftclsnofoot", not "draft", option
% should be used if it is desired that the figures are to be displayed in
% draft mode.
%
\documentclass[journal]{IEEEtran}
%
% If IEEEtran.cls has not been installed into the LaTeX system files,
% manually specify the path to it like:
% \documentclass[journal]{../sty/IEEEtran}





% Some very useful LaTeX packages include:
% (uncomment the ones you want to load)

\usepackage{graphicx}
\usepackage{amsmath}


% *** CITATION PACKAGES ***
%
%\usepackage{cite}
% cite.sty was written by Donald Arseneau
% V1.6 and later of IEEEtran pre-defines the format of the cite.sty package
% \cite{} output to follow that of IEEE. Loading the cite package will
% result in citation numbers being automatically sorted and properly
% "compressed/ranged". e.g., [1], [9], [2], [7], [5], [6] without using
% cite.sty will become [1], [2], [5]--[7], [9] using cite.sty. cite.sty's
% \cite will automatically add leading space, if needed. Use cite.sty's
% noadjust option (cite.sty V3.8 and later) if you want to turn this off
% such as if a citation ever needs to be enclosed in parenthesis.
% cite.sty is already installed on most LaTeX systems. Be sure and use
% version 4.0 (2003-05-27) and later if using hyperref.sty. cite.sty does
% not currently provide for hyperlinked citations.
% The latest version can be obtained at:
% http://www.ctan.org/tex-archive/macros/latex/contrib/cite/
% The documentation is contained in the cite.sty file itself.


% *** GRAPHICS RELATED PACKAGES ***
%
\ifCLASSINFOpdf
  % \usepackage[pdftex]{graphicx}
  % declare the path(s) where your graphic files are
  % \graphicspath{{../pdf/}{../jpeg/}}
  % and their extensions so you won't have to specify these with
  % every instance of \includegraphics
  % \DeclareGraphicsExtensions{.pdf,.jpeg,.png}
\else
  % or other class option (dvipsone, dvipdf, if not using dvips). graphicx
  % will default to the driver specified in the system graphics.cfg if no
  % driver is specified.
  % \usepackage[dvips]{graphicx}
  % declare the path(s) where your graphic files are
  % \graphicspath{{../eps/}}
  % and their extensions so you won't have to specify these with
  % every instance of \includegraphics
  % \DeclareGraphicsExtensions{.eps}
\fi
% graphicx was written by David Carlisle and Sebastian Rahtz. It is
% required if you want graphics, photos, etc. graphicx.sty is already
% installed on most LaTeX systems. The latest version and documentation
% can be obtained at: 
% http://www.ctan.org/tex-archive/macros/latex/required/graphics/
% Another good source of documentation is "Using Imported Graphics in
% LaTeX2e" by Keith Reckdahl which can be found at:
% http://www.ctan.org/tex-archive/info/epslatex/
%
% latex, and pdflatex in dvi mode, support graphics in encapsulated
% postscript (.eps) format. pdflatex in pdf mode supports graphics
% in .pdf, .jpeg, .png and .mps (metapost) formats. Users should ensure
% that all non-photo figures use a vector format (.eps, .pdf, .mps) and
% not a bitmapped formats (.jpeg, .png). IEEE frowns on bitmapped formats
% which can result in "jaggedy"/blurry rendering of lines and letters as
% well as large increases in file sizes.
%
% You can find documentation about the pdfTeX application at:
% http://www.tug.org/applications/pdftex


% *** MATH PACKAGES ***
%
%\usepackage[cmex10]{amsmath}
% A popular package from the American Mathematical Society that provides
% many useful and powerful commands for dealing with mathematics. If using
% it, be sure to load this package with the cmex10 option to ensure that
% only type 1 fonts will utilized at all point sizes. Without this option,
% it is possible that some math symbols, particularly those within
% footnotes, will be rendered in bitmap form which will result in a
% document that can not be IEEE Xplore compliant!
%
% Also, note that the amsmath package sets \interdisplaylinepenalty to 10000
% thus preventing page breaks from occurring within multiline equations. Use:
%\interdisplaylinepenalty=2500
% after loading amsmath to restore such page breaks as IEEEtran.cls normally
% does. amsmath.sty is already installed on most LaTeX systems. The latest
% version and documentation can be obtained at:
% http://www.ctan.org/tex-archive/macros/latex/required/amslatex/math/


% *** SPECIALIZED LIST PACKAGES ***
%
%\usepackage{algorithmic}
% algorithmic.sty was written by Peter Williams and Rogerio Brito.
% This package provides an algorithmic environment fo describing algorithms.
% You can use the algorithmic environment in-text or within a figure
% environment to provide for a floating algorithm. Do NOT use the algorithm
% floating environment provided by algorithm.sty (by the same authors) or
% algorithm2e.sty (by Christophe Fiorio) as IEEE does not use dedicated
% algorithm float types and packages that provide these will not provide
% correct IEEE style captions. The latest version and documentation of
% algorithmic.sty can be obtained at:
% http://www.ctan.org/tex-archive/macros/latex/contrib/algorithms/
% There is also a support site at:
% http://algorithms.berlios.de/index.html
% Also of interest may be the (relatively newer and more customizable)
% algorithmicx.sty package by Szasz Janos:
% http://www.ctan.org/tex-archive/macros/latex/contrib/algorithmicx/


% *** ALIGNMENT PACKAGES ***
%
%\usepackage{array}
% Frank Mittelbach's and David Carlisle's array.sty patches and improves
% the standard LaTeX2e array and tabular environments to provide better
% appearance and additional user controls. As the default LaTeX2e table
% generation code is lacking to the point of almost being broken with
% respect to the quality of the end results, all users are strongly
% advised to use an enhanced (at the very least that provided by array.sty)
% set of table tools. array.sty is already installed on most systems. The
% latest version and documentation can be obtained at:
% http://www.ctan.org/tex-archive/macros/latex/required/tools/


% IEEEtran contains the IEEEeqnarray family of commands that can be used to
% generate multiline equations as well as matrices, tables, etc., of high
% quality.


% *** SUBFIGURE PACKAGES ***
%\ifCLASSOPTIONcompsoc
%  \usepackage[caption=false,font=normalsize,labelfont=sf,textfont=sf]{subfig}
%\else
%  \usepackage[caption=false,font=footnotesize]{subfig}
%\fi
% subfig.sty, written by Steven Douglas Cochran, is the modern replacement
% for subfigure.sty, the latter of which is no longer maintained and is
% incompatible with some LaTeX packages including fixltx2e. However,
% subfig.sty requires and automatically loads Axel Sommerfeldt's caption.sty
% which will override IEEEtran.cls' handling of captions and this will result
% in non-IEEE style figure/table captions. To prevent this problem, be sure
% and invoke subfig.sty's "caption=false" package option (available since
% subfig.sty version 1.3, 2005/06/28) as this is will preserve IEEEtran.cls
% handling of captions.
% Note that the Computer Society format requires a larger sans serif font
% than the serif footnote size font used in traditional IEEE formatting
% and thus the need to invoke different subfig.sty package options depending
% on whether compsoc mode has been enabled.
%
% The latest version and documentation of subfig.sty can be obtained at:
% http://www.ctan.org/tex-archive/macros/latex/contrib/subfig/


% *** FLOAT PACKAGES ***
%
%\usepackage{fixltx2e}
% fixltx2e, the successor to the earlier fix2col.sty, was written by
% Frank Mittelbach and David Carlisle. This package corrects a few problems
% in the LaTeX2e kernel, the most notable of which is that in current
% LaTeX2e releases, the ordering of single and double column floats is not
% guaranteed to be preserved. Thus, an unpatched LaTeX2e can allow a
% single column figure to be placed prior to an earlier double column
% figure. The latest version and documentation can be found at:
% http://www.ctan.org/tex-archive/macros/latex/base/

%\usepackage{stfloats}
% stfloats.sty was written by Sigitas Tolusis. This package gives LaTeX2e
% the ability to do double column floats at the bottom of the page as well
% as the top. (e.g., "\begin{figure*}[!b]" is not normally possible in
% LaTeX2e). It also provides a command:
%\fnbelowfloat
% to enable the placement of footnotes below bottom floats (the standard
% LaTeX2e kernel puts them above bottom floats). This is an invasive package
% which rewrites many portions of the LaTeX2e float routines. It may not work
% with other packages that modify the LaTeX2e float routines. The latest
% version and documentation can be obtained at:
% http://www.ctan.org/tex-archive/macros/latex/contrib/sttools/
% Do not use the stfloats baselinefloat ability as IEEE does not allow
% \baselineskip to stretch. Authors submitting work to the IEEE should note
% that IEEE rarely uses double column equations and that authors should try
% to avoid such use. Do not be tempted to use the cuted.sty or midfloat.sty
% packages (also by Sigitas Tolusis) as IEEE does not format its papers in
% such ways.
% Do not attempt to use stfloats with fixltx2e as they are incompatible.
% Instead, use Morten Hogholm'a dblfloatfix which combines the features
% of both fixltx2e and stfloats:
%
% \usepackage{dblfloatfix}
% The latest version can be found at:
% http://www.ctan.org/tex-archive/macros/latex/contrib/dblfloatfix/

%\ifCLASSOPTIONcaptionsoff
%  \usepackage[nomarkers]{endfloat}
% \let\MYoriglatexcaption\caption
% \renewcommand{\caption}[2][\relax]{\MYoriglatexcaption[#2]{#2}}
%\fi
% endfloat.sty was written by James Darrell McCauley, Jeff Goldberg and 
% Axel Sommerfeldt. This package may be useful when used in conjunction with 
% IEEEtran.cls'  captionsoff option. Some IEEE journals/societies require that
% submissions have lists of figures/tables at the end of the paper and that
% figures/tables without any captions are placed on a page by themselves at
% the end of the document. If needed, the draftcls IEEEtran class option or
% \CLASSINPUTbaselinestretch interface can be used to increase the line
% spacing as well. Be sure and use the nomarkers option of endfloat to
% prevent endfloat from "marking" where the figures would have been placed
% in the text. The two hack lines of code above are a slight modification of
% that suggested by in the endfloat docs (section 8.4.1) to ensure that
% the full captions always appear in the list of figures/tables - even if
% the user used the short optional argument of \caption[]{}.
% IEEE papers do not typically make use of \caption[]'s optional argument,
% so this should not be an issue. A similar trick can be used to disable
% captions of packages such as subfig.sty that lack options to turn off
% the subcaptions:
% For subfig.sty:
% \let\MYorigsubfloat\subfloat
% \renewcommand{\subfloat}[2][\relax]{\MYorigsubfloat[]{#2}}
% However, the above trick will not work if both optional arguments of
% the \subfloat command are used. Furthermore, there needs to be a
% description of each subfigure *somewhere* and endfloat does not add
% subfigure captions to its list of figures. Thus, the best approach is to
% avoid the use of subfigure captions (many IEEE journals avoid them anyway)
% and instead reference/explain all the subfigures within the main caption.
% The latest version of endfloat.sty and its documentation can obtained at:
% http://www.ctan.org/tex-archive/macros/latex/contrib/endfloat/
%
% The IEEEtran \ifCLASSOPTIONcaptionsoff conditional can also be used
% later in the document, say, to conditionally put the References on a 
% page by themselves.


% *** PDF, URL AND HYPERLINK PACKAGES ***
%
%\usepackage{url}
% url.sty was written by Donald Arseneau. It provides better support for
% handling and breaking URLs. url.sty is already installed on most LaTeX
% systems. The latest version and documentation can be obtained at:
% http://www.ctan.org/tex-archive/macros/latex/contrib/url/
% Basically, \url{my_url_here}.




% *** Do not adjust lengths that control margins, column widths, etc. ***
% *** Do not use packages that alter fonts (such as pslatex).         ***
% There should be no need to do such things with IEEEtran.cls V1.6 and later.
% (Unless specifically asked to do so by the journal or conference you plan
% to submit to, of course. )

\begin{document}
%
% paper title
% can use linebreaks \\ within to get better formatting as desired
% Do not put math or special symbols in the title.
\title{Questionnaire based configuration of product-lines in FeatureIDE}
%
%
% author names and IEEE memberships
% note positions of commas and nonbreaking spaces ( ~ ) LaTeX will not break
% a structure at a ~ so this keeps an author's name from being broken across
% two lines.
% use \thanks{} to gain access to the first footnote area
% a separate \thanks must be used for each paragraph as LaTeX2e's \thanks
% was not built to handle multiple paragraphs
%

\author{Jens~Wiemann,~Otto-von-Guericke-Universit\"at~Magdeburg, Stephan~D\"orfler,~Otto-von-Guericke-Universit\"at~Magdeburg,\\ \{jens.wiemann,~stephan.doerfler\}@st.ovgu.de}

% note the % following the last \IEEEmembership and also \thanks - 
% these prevent an unwanted space from occurring between the last author name
% and the end of the author line. i.e., if you had this:
% 
% \author{....lastname \thanks{...} \thanks{...} }
%                     ^------------^------------^----Do not want these spaces!
%
% a space would be appended to the last name and could cause every name on that
% line to be shifted left slightly. This is one of those "LaTeX things". For
% instance, "\textbf{A} \textbf{B}" will typeset as "A B" not "AB". To get
% "AB" then you have to do: "\textbf{A}\textbf{B}"
% \thanks is no different in this regard, so shield the last } of each \thanks
% that ends a line with a % and do not let a space in before the next \thanks.
% Spaces after \IEEEmembership other than the last one are OK (and needed) as
% you are supposed to have spaces between the names. For what it is worth,
% this is a minor point as most people would not even notice if the said evil
% space somehow managed to creep in.



% The paper headers
\markboth{Otto-von-Guericke-University Magdeburg, 2016}%
{Wiemann, D\"orfler: Questionnaire based configuration of product-lines in FeatureIDE}
% The only time the second header will appear is for the odd numbered pages
% after the title page when using the twoside option.

% make the title area
\maketitle

% As a general rule, do not put math, special symbols or citations
% in the abstract or keywords.
\begin{abstract}
Variability management is an essential part of working on product lines. Feature models are an established tool to describe the set of features and constraints contained in a given software product-line, thus showing the variability. Feature models also facilitate configuration of product lines. However, for most software product lines there exists no feature model. This paper proposes a method for automatically generating feature models out of the structure of given source code and naming conventions.

Furthermore, an alternative method of configuration based on questionnaires is developed in this work. It enables users or customers to configure a product on their own. Also, it allows experts to design the questionnaires according to their domain knowledge during implementation of the product line.
\end{abstract}

% Note that keywords are not normally used for peerreview papers.
\begin{IEEEkeywords}
FeatureIDE, Feature Model, Extraction, Configuration, Questionnaire.
\end{IEEEkeywords}


\section{Introduction}
% The very first letter is a 2 line initial drop letter followed
% by the rest of the first word in caps.
% 
% form to use if the first word consists of a single letter:
% \IEEEPARstart{A}{demo} file is ....
% 
% form to use if you need the single drop letter followed by
% normal text (unknown if ever used by IEEE):
% \IEEEPARstart{A}{}demo file is ....
% 
% Some journals put the first two words in caps:
% \IEEEPARstart{T}{his demo} file is ....
% 
% Here we have the typical use of a "T" for an initial drop letter
% and "HIS" in caps to complete the first word.
<<<<<<< HEAD
\IEEEPARstart{D}{eveloping} software product lines can result in a great amount of variants, when customizing the Software to each customers needs. By developing with a feature-oriented approach the configuration of a single variant can be done by selecting the features a customer needs, automatically including its dependencies. The configuration is based on a Feature Model, that defines the available Features and its relations to one another.
=======
\IEEEPARstart{S}{oftware} product lines. foo. There are several approaches trying to control the vast amount of product variants though configuration. This allows experts to apply their domain knowledge in order to a resulting product conforming to a user's needs.
>>>>>>> origin/master

Feature Models are essential structures for the feature-oriented development and later configuration of software product lines (SPL). In such a way that they give a complete and easily understandable overview of the given features and constraints of a product-line. There are projects being developed feature-oriented, but don't have a feature model yet. This work aims at automatically generating it out of descriptive files and naming conventions, to simplify a big part of the feature model creation.

Although the feature configuration gives  Developers the ability to create custom variants, there still has to be a consultant explaining the features to the customer, trying to figure out his current and future needs. As the Software grows and gets more features, this process gets more difficult, as you can no longer explain all the features, but you still have to figure out if the customer needs them or not.

This paper considers existing methods of configuration and tries to come up with a better alternative based on questionnaires to enable users or customers to configure a product on their own and to allow experts to design the questionnaires according to their domain knowledge.

\section*{Problem statement}
Very high complexity of configuration due to many features and constraints. Domain knowledge is highly required to understand the given software product-line and being able to combine it's features to a valid configuration which satisfies the users needs.

\section*{Contribution}

Simplifying the process of configuration of product lines through extracting a feature model out of naming conventions and configuration files from an existing product line in FeatureIDE and applying a configuration wizard which guides the user though the creation of a specific product (variant) in the style of a questionnaire, thus applying the domain knowledge of an expert.

%This\cite{fpe} is\cite{qdc} a\cite{qbvm} dummy\cite{fmgp} to create citations.
% An example of a floating figure using the graphicx package.
% Note that \label must occur AFTER (or within) \caption.
% For figures, \caption should occur after the \includegraphics.
% Note that IEEEtran v1.7 and later has special internal code that
% is designed to preserve the operation of \label within \caption
% even when the captionsoff option is in effect. However, because
% of issues like this, it may be the safest practice to put all your
% \label just after \caption rather than within \caption{}.
%
% Reminder: the "draftcls" or "draftclsnofoot", not "draft", class
% option should be used if it is desired that the figures are to be
% displayed while in draft mode.
%
%\begin{figure}[!t]
%\centering
%\includegraphics[width=2.5in]{myfigure}
% where an .eps filename suffix will be assumed under latex, 
% and a .pdf suffix will be assumed for pdflatex; or what has been declared
% via \DeclareGraphicsExtensions.
%\caption{Simulation Results.}
%\label{fig_sim}
%\end{figure}

% Note that IEEE typically puts floats only at the top, even when this
% results in a large percentage of a column being occupied by floats.


% An example of a double column floating figure using two subfigures.
% (The subfig.sty package must be loaded for this to work.)
% The subfigure \label commands are set within each subfloat command,
% and the \label for the overall figure must come after \caption.
% \hfil is used as a separator to get equal spacing.
% Watch out that the combined width of all the subfigures on a 
% line do not exceed the text width or a line break will occur.
%
%\begin{figure*}[!t]
%\centering
%\subfloat[Case I]{\includegraphics[width=2.5in]{box}%
%\label{fig_first_case}}
%\hfil
%\subfloat[Case II]{\includegraphics[width=2.5in]{box}%
%\label{fig_second_case}}
%\caption{Simulation results.}
%\label{fig_sim}
%\end{figure*}
%
% Note that often IEEE papers with subfigures do not employ subfigure
% captions (using the optional argument to \subfloat[]), but instead will
% reference/describe all of them (a), (b), etc., within the main caption.


% An example of a floating table. Note that, for IEEE style tables, the 
% \caption command should come BEFORE the table. Table text will default to
% \footnotesize as IEEE normally uses this smaller font for tables.
% The \label must come after \caption as always.
%
%\begin{table}[!t]
%% increase table row spacing, adjust to taste
%\renewcommand{\arraystretch}{1.3}
% if using array.sty, it might be a good idea to tweak the value of
% \extrarowheight as needed to properly center the text within the cells
%\caption{An Example of a Table}
%\label{table_example}
%\centering
%% Some packages, such as MDW tools, offer better commands for making tables
%% than the plain LaTeX2e tabular which is used here.
%\begin{tabular}{|c||c|}
%\hline
%One & Two\\
%\hline
%Three & Four\\
%\hline
%\end{tabular}
%\end{table}


% Note that IEEE does not put floats in the very first column - or typically
% anywhere on the first page for that matter. Also, in-text middle ("here")
% positioning is not used. Most IEEE journals use top floats exclusively.
% Note that, LaTeX2e, unlike IEEE journals, places footnotes above bottom
% floats. This can be corrected via the \fnbelowfloat command of the
% stfloats package.


\section{Basics for simplification of variant configuration}
The methods and technologies we used will be listed and explained here plus references to related work for further research will be given

\subsection{FeatureIDE}

\subsection{Feature Models}

\subsection{Contraints, contradictions, SAT-solver}

\subsection{Questionnaire structures}




\section{Workflow based on an unstructured product line}
[Diagram showing the general workflow]

unstructured product line $\rightarrow$ generated feature model $\rightarrow$ optimizing (via SAT-solver?) $\rightarrow$ domain knowledge of an expert $\Rightarrow$ Questionnaire $\rightarrow$ Feature model + Questionnaire $\Rightarrow$ Product (variant)\\

Figure~\ref{img-workflow} shows the general process aimed to archive within this work. At the beginning there is only a given product line with the modular code artifacts in their specific subdirectories of the project. These code artifacts follow certain conventions to describe their integration in the context of the whole project.

Using these conventions a feature model gets extracted in which the hierarchies and dependencies of the code artifacts are included. This allows for much better overview or automated optimization, for example scanning for dead features via SAT-solver. The extracted feature model will also be used in the following step of configuration.

Configuration of product lines generally requires a good understanding of the given problems and the possible solutions to these problems as well as the specific implementations of these solutions. This domain knowledge is the most challenging part of configurations and restricts most of the possible users to do the configuration on their own.

To transfer the domain knowledge of experts to the product line a questionnaire is introduced. In the progress of implementing the product line experts also design a questionnaire in such a way that a possible user has to answer a given amount of questions to perform the configuration of a variant meeting his personal needs without him having to know all the details of the implementation of the product line or even the individual features. The two major steps of extraction and the creation and usage of the questionnaire are described in detail on the following sections.


\subsection{Extraction of a feature model}
To automatically extract a feature model out of a given software project the structure of that project must be algorithmically processable. To archive this, conventions are used for the project of which an overview follows:

\begin{itemize}
	\item All features are stored within a specified directory
	\item Each feature is stored in a separate directory
	\item A feature's parent features are stated in the name of the corresponding directory
	\begin{itemize}
		\item The complete hierarchy is displayed, except for the root feature
		\item Individual parent features are delimited with an unique symbol
	\end{itemize}
	\item Each feature directory contains a configuration file with a specified name containing:
	\begin{itemize}
		\item A descriptive name of the feature
		\item A description text for the feature
		\item All dependencies including the parent features
	\end{itemize}
\end{itemize}

All of these information are included in the resulting feature model. The most obvious correspondence lies within the hierarchy. The parent names are parsed from the directory name and -if existing- looked up in the existing partial feature model from the previously parsed features. As the directories are parsed in alphabetical order a parent feature will always be processed before it's child features.

After placing it in the correct position within the existing partial feature model each feature's configuration file gets processed. Each feature gets enriched with it's descriptive details and it's dependencies. The list of dependencies firstly gets reduced by the parent features as the feature models hierarchy already implements this kind of dependencies. The remaining dependencies are stated as constraints. To shorten the list of constraints shared dependencies between multiple features are combined to a single constraint.\\

{\color{red}TODO: State the problems:
\begin{itemize}
\item Error-Handling on failure to comply the conventions
\item Missing Parent Features $\rightarrow$ Abstract features (can be configured beforehand)
\end{itemize}}

\subsection{Questionnaire Approach}
As stated above configuration is one of the most challenging tasks within the scope of software product lines. In the previous section the automatic extraction of a feature model out of existing source code was explained. The resulting feature model yields a much better overview over the possible features and their interrelationships. Although the formalism of a feature model allows for tool support to simplify the process of configuration, still a lot of domain knowledge is required to be able to find the right combination of features for a given use-case.

This work therefore introduces a method to allow experts to apply their knowledge and understanding to a whole product line during the development and thus enabling end users to draw on this knowledge whenever a configuration is taking place.

In this work we made the decision to use a questionnaire based approach. Depending on the implementation of the questionnaire during development a partial or event complete configuration can be archived by a user through just answering the questions of the questionnaire. The concrete selection of features gets lifted to a higher level of abstraction. The user only has to decide between the possible answers presented to him in the questionnaire to best fit to his use-case. Internally the selected answers are mapped to a specified (un-)selection of features so the user avoids the hassle of considering implementation-details of each feature.

This highly depends on the implementation of the questionnaire during development. Our work therefore introduces a set of tools to easily integrate such a questionnaire. The following paragraphs will give an overview over the the possible definitions of a questionnaire.

Each page of the questionnaire is defined independently. It always contains at least a Question and more then one possible answer. The answers can be grouped analogous to the grouping of features in a feature model: \textit{OR} (at least one answer has to be selected), \textit{ALTERNATIVE} (exactly one answer has to be selected) and \textit{AND} (any number of answers can be selected).

Each answer internally has a mapping to a set of features. This set of features defines which features are selected or specifically unselected in the case of that answer being chosen by the user.

Each answer can also have an indicator to define which page of the questionnaire is to be displayed next. An answer can also indicate the end of the questionnaire. If no next page is defined the questionnaire will continue with the next page within it's definition. This allows a dynamic conditional design of the questionnaire so the user is only confronted with the exact set of questions needed to configure the variant for his specific use-case. This also allows the user to skip questions or cancel the configuration before finishing it and thus creating a partial configuration.

In this work we also introduce a data structure to hold the definition of a questionnaire. To archive easy integration we decided for a definition in XML. We defined the necessary tags to create a questionnaire which are displayed in the following code snippet:

\begin{lstlisting}
<configurationSurvey>
	<projectName>Name</projectName>
	<section id="0">
		<name>Section Name</name>
		<description>Section description</description>
	</section>
	<page id="0" sectionId="0">
		<question>Question for the user</question>
		<answers type="alternative">
			<answer nextPageId="1">
				<label>Answer label</label>
				<description>Answer description</description>
				<dependencies>
					<feature selection="false">Unselected feature</feature>
					<feature>selected feature</feature>
				</dependencies>
			</answer>
		</answers>
	</page>
</configurationSurvey>
\end{lstlisting}

\begin{tabular}{ l | l | l }
\textbf{Tag} & \textbf{attributes} & \textbf{child tags}\\
\hline
\hline
configurationSurvey & & projectName, section, page\\
\hline
section & id & name, description\\
\hline
page & id, sectionId & question, answers\\
\hline
answers & type & answer\\
\hline
answer & nextPageId & label, description, dependencies\\
\hline
dependencies & & feature\\
\hline
feature & selection & \\
\end{tabular}\\


The individual tags are explained as follows:

\begin{itemize}
\item configurationSurvey: The root tag to contain all other tags for the questionnaire.
\item section: Enables grouping of question-pages. Also displays the name and description at the top of every page.
\item page: Contains a question and the corresponding answers. Also has an indicator for a section
\item answers: Contains the individual possible answers and groups them in the specified manner.
\item answer: Defines the displayed text of an answer as well as the corresponding features. Can also have an indicator for the next page.
\item dependencies: Defines the (un-)selection of features, if the corresponding answer gets selected.
\end{itemize}

\section{Examplary scenarios}

In the fallowing chapter the workflow described in chapter~\ref{ch:workflow} is carried out based on the current project status of \textit{Odoo}\footnote{https://www.odoo.com/}. \textit{Odoo} (formerly known as OpenERP) is a suite of \textit{open-source}\footnote{https://github.com/odoo/odoo} enterprise applications, targeting companies of all sizes. As each costumer needs different functionalities from \textit{Odoo}, the product variations can be very different in size and target use. Besides them researching in the field of easing their product configuration process, this project suits our workflows requirements.

The current version of \textit{Odoo} (version 9.0, released \mbox{October 1, 2015)} has 465~contributors and over 101.200~commits. They have developed 266 features which are all inside of a top hierarchy folder named "addons". Each of them contain code as well as a configuration file named "\textunderscore\textunderscore openerp\textunderscore\textunderscore.py". Inside this file there are all the relevant informations for this feature, like dependencies, summaries , categories, versions and more descriptive informations.

At first, a feature model is generated based on the naming conventions of the project folders and the configuration files contained therein. Afterwards the feature model is corrected from any errors that occurred during the generation process. The resulting model combined with domain knowledge allows for the creation of a questionnaire that takes a minimum of choices to configure a custom tailored application.

All the \textit{Odoo} features lie inside of a folder named "addons". The code of each feature is contained in its own folder which is named like the parent feature with a suffix of its own name. According to this naming convention the feature "website\textunderscore sale" is a child of the feature "website", and parent to "website\textunderscore sale\textunderscore stock". In addition to the naming convention, there is a configuration file, written in python in each of the feature folders. These files contain a description of the feature, a name, that doesn't always match with the folder name and the dependencies that are required for this feature to work.
By first creating a list of all features based on the folder names, we can match them to the names inside of the configuration files. At this point we realize, that there are some conflicts in the naming of the 261 features. For example the feature "sale" is a standalone feature and has nothing to do with the sale from the feature "event\textunderscore sale". As the default FeatureIDE doesn't support multiple features with the same name, we decided to leave the full names with the underscores. In addition, there are some folder names where the underscore doesn't separate features, like "point\textunderscore of\textunderscore sale". For those we implemented naming exceptions where the user can add strings that represent a single feature.

We applied these steps to a total of 261 folders, containing in total more than 18.800 files, extracting 306 features. The surplus features result from abstract features, that do not have their own folder, but are implizit in the naming. The resulting feature model is arranged in a total of six hierarchical levels. Additionally, 196 constraints limit the possible valid configurations of \textit{Odoo}.


{\color{red}TODO: Fragebogen}

{\color{red}TODO: Auswertung}

%TODO: Auswertung

%Also state how some of the questions and their mapping in the questionnaire were developed.

%Finally, show the process of configuration through running a fictinal user over the questionnaire and document his decisions as well as the resulting configurations.

%TODO: test ausdenken, der alle ``Fähgkeiten'' zeigt

% Fahrradladen, Online-Shop, Social-Media, Bezahldienste, Kontakt-Portal, DE/EN, Newsletter
%
% Website builder, eCommerce, SEA
% Sales, CRM, invoicing, Point of Sale
% Inventory, Purchase
% Email Marketing, Survey

%TODO: Planung, durchführung, Auswertung

%TODO: nur ein konstruierter Fall, weil Domain Knowledge fehlt.

\section{Conclusion and Future Work}
%TODO: research questions

In this work we propose a questionnaire-based approach to support the process of configuring an existing software product line to generate an actual product. To do so, we use feature models as formalism to structure the existing code artifacts. The feature model automatically gets filled with the information we get from analyzing the structure and conventions of the existing source code. The resulting feature model contains every code artifact as a feature and also includes the constraints and restrictions.

We also propose a questionnaire to support the configuration of a product from the given product line. We introduce definitions in XML, which allow for creation of questionnaires specifically designed for configuration. A questionnaire needs to be designed carefully by the developers or experts on the domain of the product line. It allows a mapping between answers to simple questions and actual features of the product line. With this method, a user doesn't have to undergo the process of configuration in the sense of \mbox{(de-)selecting} a specific set of features. Instead, the user only has to answer questions according to its use case to do the configuration of his specific product. As the tool uses the underlying FeatureIDE, it also makes use of its functions, such as validity checks. This additionally supports the usability and correctness of the whole process.

For the exapmle of \textit{Odoo}, we were able to extract a feature model from the code. We also designed an examplary questionnaire and ran it with a simulated test user. This test user had a certain use case, which we were able to apply to the questionnaire through just answering the presented questions. As a result we got a custom tailored version of \textit{Odoo} for our test user.


% if have a single appendix:
%\appendix[Proof of the Zonklar Equations]
% or
%\appendix  % for no appendix heading
% do not use \section anymore after \appendix, only \section*
% is possibly needed

% use appendices with more than one appendix
% then use \section to start each appendix
% you must declare a \section before using any
% \subsection or using \label (\appendices by itself
% starts a section numbered zero.)
%


% Can use something like this to put references on a page
% by themselves when using endfloat and the captionsoff option.
\ifCLASSOPTIONcaptionsoff
  \newpage
\fi


% references section

% can use a bibliography generated by BibTeX as a .bbl file
% BibTeX documentation can be easily obtained at:
% http://www.ctan.org/tex-archive/biblio/bibtex/contrib/doc/
% The IEEEtran BibTeX style support page is at:
% http://www.michaelshell.org/tex/ieeetran/bibtex/
%
% argument is your BibTeX string definitions and bibliography database(s)
%\bibliography{IEEEabrv,../bib/paper}
%
% <OR> manually copy in the resultant .bbl file
% set second argument of \begin to the number of references
% (used to reserve space for the reference number labels box)
\bibliography{literature}
\bibliographystyle{IEEEtran}

% that's all folks
\end{document}


