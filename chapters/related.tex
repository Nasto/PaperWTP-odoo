

Several techniques for synthesising feature models from a set of configurations, constraints, or even publicly available product descriptions have been proposed \cite{zhang2013mining, becan2014webfml, davril2013feature}. These techniques assume and require very few informations of the source, but therefore can only create limited feature models. Furthermore they cannot be applied in our context, as they assume the availability of formal and complete descriptions of configurations and constraints. Therefore we developed a new extraction method that relies on set project structure and conventions. This leads to a feature model with a correct hierarchy, original feature names, descriptions and complete dependencies.

Davril et al. \cite{davril2013feature} presented an approach to generate a feature model out of a product description. The results from this approach equal to the information given by the folder names we are provided with, although Davril et al. lacks the hierarchy of those features. They achieve the hierarchy by mining associations. As we get the hierarchy out of the configuration files, we already have all information needed with less effort and higher probability of a correct result. However, this approach could be used in later work to handle naming conflicts or missing information.

The approach from La Rosa et al. \cite{qbvm} presents some techniques to configure a variable system based on questionnaire models and domain constraints. The main parts can be applied to feature models, although this was not the focus of their work. Therefore they only included and-groups and did not include or-groups and alternative-groups, simplifying it in a way. Nevertheless, they did present a technique to detect circular dependencies and contradictory constraints, which was not part of this work, but could be a focus for further improvements.