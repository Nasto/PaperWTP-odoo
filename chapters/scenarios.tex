In the fallowing chapter the workflow described in Chapter~\ref{ch:workflow} is carried out based on the current project status of \textit{Odoo}. At first a feature model is generated based on the naming conventions of the project folders and the configuration files contained therein. Afterwards the feature model is corrected from any errors that occurred during the generation process. The resulting model combined with domain knowledge allows for the creation of a questionnaire that takes a minimum of choices to configure a custom tailored application.

All the Odoo features lie inside of a folder named "addons". The code of each feature is contained in its own folder which is named like the parent feature with a suffix of its own name. According to this naming convention the feature "website\textunderscore sale" is a child of the feature "website", and parent to "website\textunderscore sale\textunderscore stock". In addition to the naming convention, there is a configuration file, written in python in each of the feature folders. These files contain a description of the feature, a name, that doesn't always match with the folder name and the dependencies that are required for this feature to work.
By first creating a list of all features based on the folder names, one can match them to the names inside of the configuration files. At this point we realize, that there are some conflicts in the naming of the 261 features. For example the feature "sale" is a standalone feature and has nothing to do with the sale from the feature "event\textunderscore sale". As the default FeatureIDE doesn't support multiple features with the same name, we decided to leave the full names with the underscores. In addition, there are some folder names where the underscore doesn't separate features, like "point\textunderscore of\textunderscore sale". For those we implemented naming exceptions where the user can add strings that represent a single feature.

We applied these steps to a total of 261 folders, containing in total more than 18.800 files, extracting 306 features. The surplus features result from abstract features, that do not have their own folder, but are implizit in the naming. The resulting feature model is arranged in a total of six hierarchical levels. Additionally, [D] constraints limit the possible valid configurations of \textit{Odoo}.


{\color{red}TODO: Fragebogen}

{\color{red}TODO: Auswertung}

%TODO: Auswertung

%Also state how some of the questions and their mapping in the questionnaire were developed.

%Finally, show the process of configuration through running a fictinal user over the questionnaire and document his decisions as well as the resulting configurations.

%TODO: test ausdenken, der alle ``Fähgkeiten'' zeigt

% Fahrradladen, Online-Shop, Social-Media, Bezahldienste, Kontakt-Portal, DE/EN, Newsletter
%
% Website builder, eCommerce, SEA
% Sales, CRM, invoicing, Point of Sale
% Inventory, Purchase
% Email Marketing, Survey

%TODO: Planung, durchführung, Auswertung

%TODO: nur ein konstruierter Fall, weil Domain Knowledge fehlt.