[Diagram showing the general workflow]

%TODO: Abstrakte Beschreibung des Nutzer-Szenarios

%TODO: Schaubild
unstructured product line $\rightarrow$ generated feature model $\rightarrow$ optimizing (via SAT-solver?) $\rightarrow$ domain knowledge of an expert $\Rightarrow$ Questionnaire $\rightarrow$ Feature model + Questionnaire $\Rightarrow$ Product (variant)\\

This chapter describes the general workflow we aim to archive within this work. This workflow can be seen in Figure~\ref{img-workflow}. To start off, some kind of software product line is required. It's source code has to be structured in a way that allows for an algorithm to recognize the individual features as well as their interrelations, especially the hierarchical dependencies.


The two major steps of extraction of the feature model and the creation and usage of the questionnaire are described in detail in the following sections.


\subsection{Extraction of a Feature Model}
Out of the given structure we extract the hierarchies of the features as well as additional dependencies. These are the input for the automated generation of a feature model. This allows for overview over the product line and it's variability. During our efforts to implement an automated generation of a feature model we encounter some problems for which we have to find solutions/workarounds.

The first problem consists of finding the structures of the given source code and processing them algorithmically. Most software complies to conventions regarding the naming and structuring of directories. We try to make use of these conventions. But as they are just conventions and not rules, in most source code one will find violations of the conventions. Instead of interrupting the whole process, we implement error handling. The result is a notification for the user to alert the developers about the error. Also, the remaining information, which aren't immediately affected by the error, are still extracted and used for the generation of the feature model, if possible.

Another challenge lies within the order of processing the features during the creation of the feature model. To correctly reproduce the hierarchy of the features, features have to be declared as child features of their parents. If the declared parent feature doesn't already exist, the creation of a feature model fails. So we have to ensure for each feature, that it gets processed after it's parent feature.

Another error we found is one ore more features having a parent feature which doesn't exist as a code artifact itself. Thus, the parent feature doesn't exists in the feature model and it's child features can't be placed in the hierarchy in the correct place. To still include these features in the feature model we create an abstract feature for the parent feature. The abstract feature is a direct child of the root feature. But as it doesn't have any describing code, it only has a name and no additional descriptive details.

The extracted feature model is also used in the following step of configuration. There, it visualizes the interrelations of the features and supports the understanding of the configuration steps.

\subsection{Questionnaire Approach}
Configuration is a challenging tasks within the scope of software product lines. The resulting feature model from the automated generation yields a better overview over the possible features and their interrelationships. Although the formalism of a feature model allows for tool support for the process of configuration, still domain knowledge is required to be able to find the right combination of features for a given use-case.

This work therefore introduces a method to allow experts to apply their knowledge and understanding to a whole product line during the development. Thus, users are enabled to draw on this knowledge whenever a configuration is taking place.

In this work, we made the decision to use a questionnaire-based approach. In the progress of implementation, experts also design a questionnaire. They do so in such a way, that a user has to answer a given amount of questions to perform the configuration of a product. Depending on the implementation of the questionnaire, a partial or even complete configuration can be archived by a user through just answering the questions of the questionnaire.

The selection of features gets lifted to a higher level of abstraction. The user only has to decide between the possible answers presented to him in the questionnaire to best fit to his use-case. Internally the selected answers are mapped to a specified (un-)selection of features so the user avoids the hassle of considering implementation-details of each feature. This effectively redesigns the process of configuration in such a way, that a user is independent of the knowledge about the details of the implementation and can focus on tailoring the product line to his specific use-case.

This highly depends on the implementation of the questionnaire during development. Our work therefore introduces a set of tools to easily integrate such a questionnaire. The following paragraphs will give an overview over the the possible definitions of a questionnaire.

Each page of the questionnaire is defined independently. It always contains at least a Question and more then one possible answer. The answers can be grouped analogous to the grouping of features in a feature model: \textit{or} (at least one answer has to be selected), \textit{alternative} (exactly one answer has to be selected) and \textit{and} (any number of answers can be selected).

Each answer internally has a mapping to a set of features. This set of features defines which features are selected or specifically unselected in the case of that answer being chosen by the user.

Each answer can also have an indicator to define which page of the questionnaire is to be displayed next. An answer can also indicate the end of the questionnaire. If no next page is defined the questionnaire will continue with the next page within it's definition. This allows a dynamic conditional design of the questionnaire so the user is only confronted with the exact set of questions needed to configure the variant for his specific use-case. This also allows the user to skip questions or cancel the configuration before finishing it and thus creating a partial configuration.

We also introduce a data structure to hold the definition of a questionnaire. To archive easy integration we decided for a definition in XML. We defined the necessary tags to create a questionnaire which are displayed in the following code snippet:

\begin{figure}
\begin{lstlisting}
<configurationSurvey>
	<projectName>Name</projectName>
	<section id="0">
		<name>Section Name</name>
		<description>Section description</description>
	</section>
	<page id="0" sectionId="0">
		<question>Question for the user</question>
		<answers type="alternative">
			<answer nextPageId="1">
				<label>Answer label</label>
				<description>
					Answer description
				</description>
				<dependencies>
					<feature selection="false">
						Unselected feature
					</feature>
					<feature>selected feature</feature>
				</dependencies>
			</answer>
		</answers>
	</page>
</configurationSurvey>
\end{lstlisting}
\caption{Examplary XML file for a  questionnaire}
\end{figure}

\begin{tabulary}{\linewidth}{LLL}
\textbf{Tag} & \textbf{attributes} & \textbf{child tags}\\
\hline
configurationSurvey & & projectName, section, page\\
section & id & name, description\\
page & id, sectionId & question, answers\\
answers & type & answer\\
answer & nextPageId & label, description, dependencies\\
dependencies & & feature\\
feature & selection & \\
\end{tabulary}\vspace{2.5em}

The individual tags are explained as follows:

\begin{itemize}
\item \texttt{configurationSurvey}: The root tag to contain all other tags for the questionnaire.
\item \texttt{section}: Enables grouping of question-pages. Also displays the name and description at the top of every page.
\item \texttt{page}: Contains a question and the corresponding answers. Also has an indicator for a section
\item \texttt{answers}: Contains the individual possible answers and groups them in the specified manner.
\item \texttt{answer}: Defines the displayed text of an answer as well as the corresponding features. Can also have an indicator for the next page.
\item \texttt{dependencies}: Defines the (un-)selection of features, if the corresponding answer gets selected.
\end{itemize}

%TODO: Am ende kommt ein konfiguriertes Produkt raus.