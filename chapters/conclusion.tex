%TODO: research questions

{\color{red}Ihr geht hier in die falsche Richtung. Schreibt einfach was jetzt geht, ihr bracht hier nicht zu diskutieren was notwendig ist oder so (wie z.b. im letzten absatz)}

In this work we propose a questionnaire-based approach to support the process of configuring an existing software product line to generate an actual product. To do so, we use feature models as formalism to structure the existing code artifacts. The feature model automatically gets filled with the information we get from analyzing the structure and conventions of the existing source code. The resulting feature model contains every code artifact as a feature and also includes the constraints and restrictions.

We also propose a questionnaire to support the configuration of a product from the given product line. We introduce definitions in XML, which allow for creation of questionnaires specifically designed for configuration. A questionnaire needs to be designed carefully by the developers or experts on the domain of the product line. It allows a mapping between answers to simple questions and actual features of the product line. With this method, a user doesn't have to undergo the process of configuration in the sense of \mbox{(de-)selecting} a specific set of features. Instead, the user only has to answer questions according to its use case to do the configuration of his specific product. As the tool uses the underlying FeatureIDE, it also makes use of its functions, such as validity checks. This additionally supports the usability and correctness of the whole process.

For the exapmle of \textit{Odoo}, we were able to extract a feature model from the code. We also designed an examplary questionnaire and ran it with a simulated test user. This test user had a certain use case, which we were able to apply to the questionnaire through just answering the presented questions. As a result we got a custom tailored version of \textit{Odoo} for our test user.