% The very first letter is a 2 line initial drop letter followed
% by the rest of the first word in caps.
% 
% form to use if the first word consists of a single letter:
% \IEEEPARstart{A}{demo} file is ....
% 
% form to use if you need the single drop letter followed by
% normal text (unknown if ever used by IEEE):
% \IEEEPARstart{A}{}demo file is ....
% 
% Some journals put the first two words in caps:
% \IEEEPARstart{T}{his demo} file is ....
% 
% Here we have the typical use of a "T" for an initial drop letter
% and "HIS" in caps to complete the first word.
<<<<<<< HEAD
\IEEEPARstart{D}{eveloping} software product lines can result in a great amount of variants, when customizing the Software to each customers needs. By developing with a feature-oriented approach the configuration of a single variant can be done by selecting the features a customer needs, automatically including its dependencies. The configuration is based on a Feature Model, that defines the available Features and its relations to one another.
=======
\IEEEPARstart{S}{oftware} product lines. foo. There are several approaches trying to control the vast amount of product variants though configuration. This allows experts to apply their domain knowledge in order to a resulting product conforming to a user's needs.
>>>>>>> origin/master

Feature Models are essential structures for the feature-oriented development and later configuration of software product lines (SPL). In such a way that they give a complete and easily understandable overview of the given features and constraints of a product-line. There are projects being developed feature-oriented, but don't have a feature model yet. This work aims at automatically generating it out of descriptive files and naming conventions, to simplify a big part of the feature model creation.

Although the feature configuration gives  Developers the ability to create custom variants, there still has to be a consultant explaining the features to the customer, trying to figure out his current and future needs. As the Software grows and gets more features, this process gets more difficult, as you can no longer explain all the features, but you still have to figure out if the customer needs them or not.

This paper considers existing methods of configuration and tries to come up with a better alternative based on questionnaires to enable users or customers to configure a product on their own and to allow experts to design the questionnaires according to their domain knowledge.

\section*{Problem statement}
Very high complexity of configuration due to many features and constraints. Domain knowledge is highly required to understand the given software product-line and being able to combine it's features to a valid configuration which satisfies the users needs.

\section*{Contribution}

Simplifying the process of configuration of product lines through extracting a feature model out of naming conventions and configuration files from an existing product line in FeatureIDE and applying a configuration wizard which guides the user though the creation of a specific product (variant) in the style of a questionnaire, thus applying the domain knowledge of an expert.

%This\cite{fpe} is\cite{qdc} a\cite{qbvm} dummy\cite{fmgp} to create citations.