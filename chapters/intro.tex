% The very first letter is a 2 line initial drop letter followed
% by the rest of the first word in caps.
% 
% form to use if the first word consists of a single letter:
% \IEEEPARstart{A}{demo} file is ....
% 
% form to use if you need the single drop letter followed by
% normal text (unknown if ever used by IEEE):
% \IEEEPARstart{A}{}demo file is ....
% 
% Some journals put the first two words in caps:
% \IEEEPARstart{T}{his demo} file is ....
% 
% Here we have the typical use of a "T" for an initial drop letter
% and "HIS" in caps to complete the first word.
Developing feature oriented, although bringing with it some overhead, facilitates the creation of software when it is planned on custom tailoring it for a multiple number of clients. Feature oriented means developing and maintaining single features that can be combined with others to create a whole product. All of the individual features make up a product line, whereas a subset of those features create a variant of this product. Developing software product lines can result in a large amount of variants, when customizing the software to each customers needs. By developing with a feature-oriented approach the configuration of a single variant can be done by selecting the features a customer needs, automatically including its dependencies. The configuration is based on a Feature Model, that defines the available Features and its relations to one another.

% TODO: fosd buch zitieren, FM Zitat
Feature Models are essential structures for the feature-oriented development and later configuration of software product lines (SPL). They give a complete and understandable overview of the given features and constraints of a product-line. There are projects being developed in a feature-oriented manner, but don't have a feature model yet. Implementing it on top of a given Project can result in a complex task due to the amount of its features and the constraints between them. Nevertheless the benefits of a correct feature model justify the effort to extract one out of a live project. This work aims at automatically generating it out of descriptive files and naming conventions, to simplify a big part of the feature model creation.

Although the feature configuration gives developers the ability to create custom variants, there still has to be a consultant explaining the features to the customer, trying to figure out his current and future needs. As the software grows and gets more features, this process gets difficult, as one can no longer explain all the features, but you still have to figure out if the customer needs them or not. This paper considers existing methods of configuration and tries to come up with a better alternative based on questionnaires to enable users or customers to configure a product on their own and to allow experts to design the questionnaires according to their domain knowledge.
 
FeatureIDE is an open source IDE for feature-oriented-software development. It provides all the functionality needed to programmatically generate and work with feature models.
% TODO: was benutzt ihr denn genau?
 The underlying Eclipse enables us to implement the feature extraction and the questionnaire as a plugin.

This work demonstrates the feature model extraction and questionnaire creation based on a real life Project. The Project, an open source ERP system called \textit{Odoo}\footnote{https://www.odoo.com/} is predestined to be developed feature-oriented due to a huge amount of features and complex constraints. It is currently being developed and structured feature oriented, although it doest't contain a feature model yet. Furthermore, \textit{Odoo} reached a critical amount of features where a salesman cannot consult a customer by going through all of the features anymore, but has to come up with a more effective way.

Contributions:
%TODO: warum? -> jeweils Begründungen
\begin{itemize}
\item Automatic feature model generation based on descriptive files and naming conventions
\item Eclipse plugin, that enables the creation of a questionnaire based on a configuration file and the associated feature model
\item The questionnaire changes as choices are made based on the logic behind the configuration file and the dependencies of the feature model
\item Exiting the questionnaire always results in a valid configured variant 
\end{itemize}