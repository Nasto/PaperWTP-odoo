% The very first letter is a 2 line initial drop letter followed
% by the rest of the first word in caps.
% 
% form to use if the first word consists of a single letter:
% \IEEEPARstart{A}{demo} file is ....
% 
% form to use if you need the single drop letter followed by
% normal text (unknown if ever used by IEEE):
% \IEEEPARstart{A}{}demo file is ....
% 
% Some journals put the first two words in caps:
% \IEEEPARstart{T}{his demo} file is ....
% 
% Here we have the typical use of a "T" for an initial drop letter
% and "HIS" in caps to complete the first word.
\IEEEPARstart{F}{eature} Modell sind als Werkzeug f\"ur Produktilinien unverzichtbar. Daher sollen sie in dieser Arbeit automatisch generiert werden, aus Konventionen, die in dem Projekt KONSEQUENT *hust* umgesetzt wurden.
Auch die Konfiguration ist ein elementarer Bestandteil der Arbeit mit Produktlinien, daher soll auch sie in dieser Arbeit Beachtung finden. Dazu werden zun\"achst bestehende Methoden betrachtet und miteinander verglichen. Dem wird ein neuer Ansatz basierend auf Frageb\"ogen gegen\"ubergestellt. F\"ur diesen Ansatz werden zun\"achst die Anforderungen an ein Konfigurationssystem ermittelt. Auch werden die zu erreichenden Verbesserungen gegen\"uber den bestehenden Systemen vorgestellt, inbesondere im Bereich der \textit{Usability}. Anhand von vorangegangenen Untersuchungen zur Struktur von Frageb\"ogen wird anschlie\ss end ein Konzept entwickelt, das es Spezialisten erlaubt, ihre Dom\"anenkenntnisse in die Konfiguration einflie\ss en zu lassen, insbesondere in Bezug auf die Reihenfolge der zu treffenden Entscheidungen im Rahmen einer Konfiguration einer Produktlinie.

Abschlie\ss end wird das erarbeitete Konzept anhand einer beispielhaften Implementierung in die \textit{featureIDE} untersucht und insbesondere unter Anbetracht der eingangs ermittelten Anforderungen bewertet.