In diesem Paper soll beschrieben werden, wie die Generierung eines Feature-Modells aus der Ordnerstruktur und den Konfigurationsdateien einer Produktlinie vorgenommen wird, hier am Beispiel von odoo.
Weiterhin wird der Vorgang der Konfiguration von Produktlinien betrachtet. Dazu wird zun\"achst untersucht, welche bestehenden Verfahren verwendet werden und ein Konzept, aufbauend auf dem Prinzip eines Fragebogens, entwickelt. Dabei soll auch die Erstellung eines solchen Fragebogens und die Beachtung einer eventuell gew\"unschten Reihenfolge in die Entwicklung des Konzeptes einflie\ss en.
Abschlie\ss end wird eine beispielhafte Umsetzung dieses Konzeptes in der IDE \textit{featureIDE} mit den zuvor ermittelten Anforderungen abgeglichen und bewertet.